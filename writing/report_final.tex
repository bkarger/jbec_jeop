\documentclass[]{article}
\usepackage{lmodern}
\usepackage{amssymb,amsmath}
\usepackage{ifxetex,ifluatex}
\usepackage{fixltx2e} % provides \textsubscript
\ifnum 0\ifxetex 1\fi\ifluatex 1\fi=0 % if pdftex
  \usepackage[T1]{fontenc}
  \usepackage[utf8]{inputenc}
\else % if luatex or xelatex
  \ifxetex
    \usepackage{mathspec}
  \else
    \usepackage{fontspec}
  \fi
  \defaultfontfeatures{Ligatures=TeX,Scale=MatchLowercase}
\fi
% use upquote if available, for straight quotes in verbatim environments
\IfFileExists{upquote.sty}{\usepackage{upquote}}{}
% use microtype if available
\IfFileExists{microtype.sty}{%
\usepackage{microtype}
\UseMicrotypeSet[protrusion]{basicmath} % disable protrusion for tt fonts
}{}
\usepackage[margin=1in]{geometry}
\usepackage{hyperref}
\hypersetup{unicode=true,
            pdfauthor={Jacob Fontenot, Burton Karger, Eric Wendt, Caroline Wendt},
            pdfborder={0 0 0},
            breaklinks=true}
\urlstyle{same}  % don't use monospace font for urls
\usepackage{graphicx,grffile}
\makeatletter
\def\maxwidth{\ifdim\Gin@nat@width>\linewidth\linewidth\else\Gin@nat@width\fi}
\def\maxheight{\ifdim\Gin@nat@height>\textheight\textheight\else\Gin@nat@height\fi}
\makeatother
% Scale images if necessary, so that they will not overflow the page
% margins by default, and it is still possible to overwrite the defaults
% using explicit options in \includegraphics[width, height, ...]{}
\setkeys{Gin}{width=\maxwidth,height=\maxheight,keepaspectratio}
\IfFileExists{parskip.sty}{%
\usepackage{parskip}
}{% else
\setlength{\parindent}{0pt}
\setlength{\parskip}{6pt plus 2pt minus 1pt}
}
\setlength{\emergencystretch}{3em}  % prevent overfull lines
\providecommand{\tightlist}{%
  \setlength{\itemsep}{0pt}\setlength{\parskip}{0pt}}
\setcounter{secnumdepth}{0}
% Redefines (sub)paragraphs to behave more like sections
\ifx\paragraph\undefined\else
\let\oldparagraph\paragraph
\renewcommand{\paragraph}[1]{\oldparagraph{#1}\mbox{}}
\fi
\ifx\subparagraph\undefined\else
\let\oldsubparagraph\subparagraph
\renewcommand{\subparagraph}[1]{\oldsubparagraph{#1}\mbox{}}
\fi

%%% Use protect on footnotes to avoid problems with footnotes in titles
\let\rmarkdownfootnote\footnote%
\def\footnote{\protect\rmarkdownfootnote}

%%% Change title format to be more compact
\usepackage{titling}

% Create subtitle command for use in maketitle
\providecommand{\subtitle}[1]{
  \posttitle{
    \begin{center}\large#1\end{center}
    }
}

\setlength{\droptitle}{-2em}

  \title{The most difficult school subject\\
(according to Jeopardy)}
    \pretitle{\vspace{\droptitle}\centering\huge}
  \posttitle{\par}
  \subtitle{ERHS 535}
  \author{Jacob Fontenot, Burton Karger, Eric Wendt, Caroline Wendt}
    \preauthor{\centering\large\emph}
  \postauthor{\par}
      \predate{\centering\large\emph}
  \postdate{\par}
    \date{12/13/2019}


\begin{document}
\maketitle

\textbf{The final report should not exceed 1,500 words. You should aim
for no more than three figures and tables.}

\hypertarget{part-i}{%
\section{Part I}\label{part-i}}

\hypertarget{research-question}{%
\subsection{Research question}\label{research-question}}

What are the most important and most difficult school subjects according
to Jeopardy!?

\hypertarget{introduction}{%
\subsection{Introduction}\label{introduction}}

We initially sorted the Jeopardy! data set according to the most
commonly asked questions in the history of the show. We found that
``What is Australia?'' was the most commonly occurring question in the
complete Jeopardy! data set. The next 32 most commonly occurring
questions were also geography related, indicating knowledge of geography
is critical to success in Jeopardy!. This prompted us to consider how a
jeopardy-based education would differ from a public school education.
Here we investigated if school subjects deemed by us (and students more
generally) to be the most important and the most difficult, are also the
most important and difficult according to the history of Jeopardy!.

\hypertarget{methods}{%
\subsection{Methods}\label{methods}}

To evaluate the most difficult and important school subjects according
to Jeopardy!, we created a dataframe of school subjects with lists of
terms related to those subjects. We obtained these vocab terms from a
commonly used online learning platform that provides study tools for
various school subjects. We chose to divide the Jeopardy! data set
according to science, mathematics, history, english, geography, and art
related questions. We matched vocab terms from our vocab dataframe with
questions in the Jeopardy! dataframe in order to filter the large
Jeopardy! dataframe into a dataframe limited to Jeopardy! questions that
matched matched our vocab terms.

To facilitate joining the dataframes, we converted all Jeopardy!
questions and school vocab terms into words exclusively limited to
lowercase letters with all special characters removed. To account for
differences in the number of vocab terms in each subject, before joining
the data frames, we randomly sampled 155 vocab terms from each subject.
We created a script to match the ``question'' from the Jeopardy!
dataframe to the corresponding key term in the subject data frame. The
script expects the standard Jeopardy! dataframe obtained from the
Jeopardy! archive as well as a key terms dataframe for a given subject.
There should be a subject column with the subject (e.g.,
\texttt{science}) listed for each row next to each vocab term. This
column is what is later used to group the Jeopardy! questions by
subject. We used \texttt{inner\_join} from the \texttt{dplyr} package to
perform matching on the Jeopardy! questions.

We discovered that the structure of the game show changed with regard to
round and value after November 26, 2001. That is, values in round one
and two doubled their previous values after this date. For instance, the
first round value originally ranged from 100 to 500; however, the first
round value was doubled to 200 dollars to 1,000 dollars after November
26, 2001. Therefore, we separated the data into two categories to
delinate which observations adhered to the particular round-value
frameworks before and after this date. Further, with the joined
dataframe in place, we calculated the number of occurrences of each
school subject as well as the average value of clues containing each
subject for each of the two date ranges. We defined the most important
school subject according to Jeopardy! as the topic with the most
occurrences, and the most difficult as the subject with the highest
average monetary value. We graphed these results using an interactive
heat map facetd by both round (e.g., 1 or 2) and date (e.g., before or
after November 26, 2001).

\hypertarget{results}{%
\subsection{Results}\label{results}}

\begin{figure}
\centering
\includegraphics{/Users/burtonkarger/Documents/jbec_jeop/graphs/hmap_dd.png}
\caption{``Heat Map Daily Doubles''}
\end{figure}

\includegraphics{/Users/burtonkarger/Documents/jbec_jeop/graphs/hmap_4.png}
\includegraphics{/Users/burtonkarger/Documents/jbec_jeop/graphs/hmap_1.png}

\textbf{include plots and tables here see}
\texttt{plots\_caroline\_final.Rmd}

most important subject most difficult subject

\hypertarget{conclusions}{%
\subsection{Conclusions}\label{conclusions}}

So what? How do your results compare with what other people have found
out about your research question? Based on what you found, are there now
other things you want to check out?

\hypertarget{part-ii-tutorial---for-slides}{%
\section{Part II: Tutorial - For
Slides?}\label{part-ii-tutorial---for-slides}}

Overview of your approach in R: Step us through a condensed version of
how you did your project

\hypertarget{interesting-packagestechniques}{%
\subsection{Interesting
packages/techniques}\label{interesting-packagestechniques}}

Spend a bit more time on any parts that you found particularly
interesting or exciting. Were there packages you used that were helpful
that we haven't talked about in class? Did you find out how to do
anything that you think other students could use in the future? Did you
end up writing a lot of functions to use? Did you have an interesting
way of sharing code and data among your group members?

\hypertarget{lessons-learned}{%
\section{Lessons learned}\label{lessons-learned}}

If you were to do this project again from scratch, what would you do
differently? Were there any big wrong turns along the way? Did you find
out how to do something late in the project that would have saved you
time if you'd started using it earlier?

\hypertarget{references}{%
\section{References}\label{references}}

Quizlet Inc.~(2019). Quizlet. Retrieved from \url{https://quizlet.com/}


\end{document}
